\documentclass[twoside]{article}
\usepackage{amssymb}
\usepackage{amsthm}
\usepackage{amsmath}
\usepackage{amsfonts}
\usepackage[utf8]{inputenc}
\usepackage[spanish]{babel}
\usepackage{tikz}
\usepackage{centernot}
\usepackage{hyperref}
\usepackage{fancyhdr}
\usepackage{lipsum}
\usepackage{subcaption}
\hypersetup{
    colorlinks,
    citecolor=black,
    filecolor=black,
    linkcolor=black,
    urlcolor=black
}
\usepackage{xurl}
\usepackage[top=1in, bottom=1.5in, left=1in, right=1in]{geometry}
\pagestyle{fancy}
\fancyhead{}
\fancyhead[L]{\leftmark}
\fancyfoot{}
\fancyfoot[C]{\thepage}
\newcommand{\enquote}[1]{``#1''}
\usepackage{float}
\usepackage[parfill]{parskip}
\newcommand{\image}[2]{
\begin{figure}[H]
    \includegraphics[width=#1 cm]{../images/#2.png}
    \centering
\end{figure}
}

\title{Trabajo CUIA: Propuesta}
\author{XuSheng Zheng}
\date{}

\begin{document}

\maketitle
\section{Descripción general de la aplicación}
Se trata de una aplicación que servirá como cambiador de ropa virtual. La plataforma elegida es Windows con lenguajes de desarrollo Python y C\#.

En primer lugar se modelará el cuerpo del usuario mediante las imágenes obtenidas a partir de una webcam. En segundo lugar se construirán modelos de ropa a elegir por el usuario sobre el modelo de la persona y se proyectarán en la pantalla. El usuario podrá modificar la talla o cambiar de ropa mediante peticiones. 

El objetivo de esta aplicación es reducir las devoluciones de las compras online debido al desconocimiento sobre los tamaños reales de los artículos.

\section{Papel que juegan las tecnologías}
\subsection{Reconocimiento e identificación de imágenes}
La aplicación reconocerá el cuerpo del usuario generando un modelo adaptado mediante el paquete cvzone de Python y utilidades de Unity.
\subsection{Procesado de lenguaje natural}
El usuario podrá solicitar cambiar de talla o de ropa mediante voz.
\subsection{Realidad aumentada}
La aplicación generará modelos virtuales de ropa con diferentes tallas a eligir sobre el modelo del usuario y los proyectará sobre la imagen captada por la cámara.

\end{document}
